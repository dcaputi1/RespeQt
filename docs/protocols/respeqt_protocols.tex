\documentclass[10pt]{article}
\usepackage[utf8]{inputenc}
\usepackage{a4wide}
\usepackage{graphicx}

%%\usepackage{natbib}

\title{RespeQt protocols}
\author{Jochen Schäfer (JoSch)}
\date{May 2023}

\begin{document}
\maketitle
\tableofcontents

\part{Introduction}
This article documents the protocols user by RespeQt and SpartaDos X to communicate with each other.

\part{RespeQt Client (RCl)}
\section{Introduction}
RCl always uses the device id 0x46 (P7: ?) to communicate. 

\section{Command overview}
\begin{tabular}[h!]{l|l|l|l}
Cmd & Name & Parameters & Return value \\ \hline 
0x93 & Send datetime & None & Date and Time \\
0x94 & Swap disks & Aux1 = slot\#, Aux2 = slot\# & Success or failure \\
0x95 & Unmount disk & Aux1 = slot\# & Success or failure \\
0x96 & Mount disk & Aux1 = slot\# & Success or failure \\
0x97 & Create and mount image & cf. description & Success or failure \\
0x98 & Toggle auto commit & None & Success or failure 
\end{tabular}

\section{Details}
This section details all the commands used in RCl

\subsection{0x93 - Send date and time}
This command requests the date and time from the server.
The client then sets the date and time on the Atari.

\paragraph{Parameters}
None

\paragraph{Return value}
The server returns a dataframe containing 6 bytes with the recent date and time as follows:\\

\begin{tabular}{l|l|l}
Index & Size & Meaning \\ \hline
0 & Byte & Day of month \\
1 & Byte & Month \\
2 & Byte & Year without century \\
3 & Byte & Hours \\
4 & Byte & Minutes \\
5 & Byte & Seconds
\end{tabular}

\subsection{0x94 - Swap disk images}
This command swaps two images in their places

\paragraph{Parameters}
Aux 1 and Aux 2 contain the respective slot indices of the disk images to swap. Their value is always between 1 and 9 or between 26 and 31.

\paragraph{Return value}
If one of the image indices is out of range, the server sends a NAK.
If the operation is successful, the server sends an ACK.

\subsection{0x95 - Umount disk images}
This command unmount one or all disk images

\paragraph{Parameters}
Aux 1 contains the slot index of the disk image to be unmounted. The value is always between 1 and 9 or between 26 and 31 or is -6 / 249. If the slot index is -6 all disk images are unmounted.

\paragraph{Return value}
If the slot index is out of range, the server sends a NAK.
If the specified image or in case of index -6 any image has to be saved, then nothing gets unmounted and the server sends a NAK.
If the operation is successful, then the server sends an ACK.

\subsection{0x96 - Mount disk image}
This command mounts a disk image in the next free slot.

\paragraph{Parameters}
If Aux 1 is 0, the server mounts the image with the file name sent in the following data frame in the next free slot.
If Aux 1 is not 0, then the slot index of the last mount is requested. 

\paragraph{Return value}
If Aux 1 is 0, the server sends a NAK, if the mount was not possible, otherwise the server sends an ACK. The filename has to be in 8.3 format.
If Aux 2 is not 0, the server returns an ACK, followed by a data frame of size 1 Byte, containing the slot index.

\subsection{0x97 - Create and mount disk image}
This command creates a disk image as specified and then mounts it in the next free slot

\paragraph{Parameters}
If Aux 1 is 0, the server mounts the image with the file name sent in the following data frame in the next free slot. The filename has to be in 8.3 format. The requested image format is appended to the filename by '.' and the type byte.
If Aux 1 is not 0, then the slot index of the last mount is requested.\\

\begin{tabular}{ll}
Type number & Disk format \\ \hline
1 & SSSD \\
2 & SSES \\
3 & SSDD \\
4 & DSDD \\
5 & DDHD \\
6 & QDHD
\end{tabular}


\paragraph{Return value}
If Aux 1 is 0, the server sends a NAK, if the mount was not possible, otherwise the server sends an ACK. The filename has to be in 8.3 format.
If Aux 2 is not 0, the server returns an ACK, followed by a data frame of size 1 Byte, containing the slot index.

\subsection{0x98 - Toggle auto commit}
This command toggles the auto commit flag.

\paragraph{Parameters}
Aux 1 contains the slot index of the auto commit flag to be toggled. The value is always between 1 and 9 or between 26 and 31 or is -6 / 249. If the slot index is -6 the auto commit flag of all slots are toggled.

\paragraph{Return value}
If the slot index is out of range, the server sends a NAK.
If the operation is successful, then the server sends an ACK.

\section{Proposals}
This sections details proposals for new commands to be used in RCl.

\part{PCLINK}
\section{Introduction}
PCLINK always uses the device id 0x6F (XXX: ?) to communicate. The protocol used by PCLINK is DOS2DOS (cf {\em BIBREF to DOS2DOS website})

\section{Command overview}
\begin{tabular}[h!]{l|l|l|l}
Cmd & Name & Parameters & Return value \\ \hline
0x00 & FREAD & & data frame with read data \\
0x01 & FWRITE & & success or failure \\
0x02 & FSEEK & & success or failure \\
0x03 & FTELL & & file position \\
0x04 & FLEN & & file length \\
0x06 & FNEXT & & \\
0x07 & FCLOSE & & success or failure\\
0x08 & INIT & & success or failure \\
0x09 & FOPEN & & success or failure \\
0x0A & FFIRST & & first directory entry \\
0x0B & RENAME & & success or failure \\
0x0C & REMOVE & & success or failure \\
0x0D & CHMOD & & success or failure \\
0x0E & MKDIR & & success or failure \\
0x0F & RMDIR & & success or failure \\
0x10 & CHDIR & & success or failure \\
0x11 & GETCWD & & current working directory \\
0x13 & DFREE & & success or failure \\
0x14 & CHVOL & & success or failure 
\end{tabular}

\section{Protocol sequence}
On the SIO protocol layer, the commands as described above are wrapped into the following commands:

\begin{tabular}[h!]{l|l|l|l}
Cmd & Name & Parameters & Return value \\ \hline 
? 0x3F & Get Hi-Speed & None & Hi-Speed setting \\
P 0x50 & Begin subcommand & Aux1 = Buffer size, Aux 2 = protocol version, device \# & Success or failure \\
R 0x52 & End subcommand & Aux1 = Buffer size, Aux 2 = protocol version, device \# & Success or failure \\
S 0x53 & Get Status & None & Status 
\end{tabular}\\

Every command begins with sending a P with the destination unit number (0-15) in the lower 4 bits of AUX2. Upon command ACK the client sends a data frame, which contains the command byte and its respective parameters (cf \ref{PCLINK:PARBUF}). The size of this data frame has to match the value sent in AUX 1 of the P command, because not all commands are using the complete parameter structure. If an error occurs, the server records the error.

The client then sends a status request to the server to get the status of the addressed device.
The server also sends a possible correction of the parameter structure size for adoption by the client. If errors occured, the client is able to abort the operation.

The client ends the sub command by sending an R to the addressed device.

If a P is send to a device before the preceding sub command is ended with R, the server is sending an command NAK. 


\section{Data structures}
\subsection{ Parameter buffer }\label{PCLINK:PARBUF}
All sub commands use the following structure to submit the parameters to their respective sub command. The client chooses how much of the structure is needed by supplying a structure size. The server can correct the client on the structure size needed for the operation. 

\begin{tabular}{l|l|l}
name & type & function \\ \hline
fno & byte & sub command number \\
handle & byte & handle to opened file or directory \\
f1, f2, f3, f4, f5, f6 & byte & general purpose bytes \\
fmode & byte & Open file mode, cf \ref{PCLINK:FMODE} \\
fatr1 & byte & Search file mask, cf \ref{PCLINK:FATR1} \\
fatr2 & byte & Transferred file attributes, cf \ref{PCLINK:FATR2} \\
name & string of length 12 & File or directory name \\
names & string of length 12 & File or directory name \\
path & string of length 65 & Path to file or directory
\end{tabular}

\subsection{fatr1}\label{PCLINK:FATR1}
This bit mask selects the attributes, when files are searched. Value 0x00 selects all files.

\begin{tabular}{l|l|l}
Value & Name & Meaning \\ \hline
0x01 & RA\_PROTECT & Protected files \\
0x02 & RA\_HIDDEN & Hidden files \\
0x04 & RA\_ARCHIVED & Files with the Archive attribute set \\
0x08 & RA\_SUBDIR & Sub directories \\
0x10 & RA\_NO\_PROTECT & Not protected files \\
0x20 & RA\_NO\_HIDDEN & Not hidden files \\
0x40 & RA\_NO\_ARCHIVED & File without the Archive attribute set \\
0x80 & RA\_NO\_SUBDIR & Not sub directories
\end{tabular}

\subsection{fatr2}\label{PCLINK:FATR2}
This bit mask repesent the transferred file attributes to be set.

\begin{tabular}{l|l|l}
Value & Name & Meaning \\ \hline
0x01 & SA\_PROTECT & Protect file \\
0x02 & SA\_HIDE & Hide file \\
0x04 & SA\_ARCHIVE & Set Archive flag on file \\
0x10 & SA\_UNPROTECT & Unprotect file \\
0x20 & SA\_UNHIDE & Unhide file \\
0x40 & RA\_UNARCHIVE & Remove Archive flag on file
\end{tabular}

\subsection{fmode}\label{PCLINK:FMODE}
\begin{tabular}{l|l}
Value & Meaning \\ \hline
0x04 & Read \\
0x08 & Write \\
0x09 & Append  \\
0x0C & Read and Write
\end{tabular}\footnote{Obviously, 0x09 really uses Bit 0, which is used as Append flag. Should we only record the value making sense or the bit masks? 0x0D is plausible for Read and Write with appending}



\section{Details}
\subsection{0x00 - FREAD Read from file or directory }
This sub command reads from a file or directory, which was opened for the requested unit before. If no file was opened before, the server answers with a command NAK.
If the opened handle is to a directory, the directory is read (\marginpar{direcotry read means what?}).
If the read was successful, the server answers with a command ACK.

\paragraph{Parameters}
Parameter buffer, cf. \ref{PCLINK:PARBUF}

\paragraph{Return value}

\subsection{0x01 - FWRITE Write to file }
This sub command writes to a file or directory, which was opened for the requested unit before. If no file or directory was opened before, the server answers with a command NAK.
If the opened handle is to a directory, the operation is ignored.
If the write was successful, the server answers with a command ACK.

\paragraph{Parameters}
Parameter buffer, cf. \ref{PCLINK:PARBUF}

\paragraph{Return value}

\subsection{0x02 - FSEEK Seek to position in file }
This sub command sets the position of a file or directory, which was opened for the requested unit before. If no file or directory was opened before, the server answers with a command NAK.
If setting the position was successful, the server answers with a command ACK.

\paragraph{Parameters}
Parameter buffer, cf. \ref{PCLINK:PARBUF}

\paragraph{Return value}

\subsection{0x03 - FTELL Return the position in file }
This sub command returns the current position of the handle to a file or directory.
If no file or directory was opened before, the server answers with a command NAK.
If setting the position was successful, the server answers with a command ACK.

\paragraph{Parameters}
Parameter buffer, cf. \ref{PCLINK:PARBUF}

\paragraph{Return value}
The position into the file or directory is returned in 3 bytes, starting with the lowest bits first. \marginpar{better wording}

\subsection{0x04 - FLEN Return file length }
This sub command returns the current size of the handle to a file or directory.
If no file or directory was opened before, the server answers with a command NAK.
If setting the position was successful, the server answers with a command ACK.

\paragraph{Parameters}
Parameter buffer, cf. \ref{PCLINK:PARBUF}
\paragraph{Return value}
The position into the file or directory is returned in 3 bytes, starting with the lowest bits first. (\marginpar{better wording})

\subsection{0x06 - FNEXT Read next directory entry }
This sub command returns the next directory entry.
If no file or directory was opened before, the server answers with a command NAK.
If setting the position was successful, the server answers with a command ACK.

\paragraph{Parameters}
Parameter buffer, cf. \ref{PCLINK:PARBUF}

\paragraph{Return value}
The server sends back a data frame beginning with the handle (1 byte), followed by the directory entry (25 bytes).

\subsection{0x07 - FCLOSE Close the file handle }
\paragraph{Parameters}
Parameter buffer, cf. \ref{PCLINK:PARBUF}

\paragraph{Return value}

\subsection{0x08 - INIT Initialize unit data structures }
\paragraph{Parameters}
Parameter buffer, cf. \ref{PCLINK:PARBUF}

\paragraph{Return value}

\subsection{0x09 - FOPEN Open file handle }
\paragraph{Parameters}
Parameter buffer, cf. \ref{PCLINK:PARBUF}

\paragraph{Return value}

\subsection{0x0A - FFIRST Read first directory entry }
\paragraph{Parameters}
Parameter buffer, cf. \ref{PCLINK:PARBUF}

\paragraph{Return value}

\subsection{0x0B - RENAME Rename file or directory }
\paragraph{Parameters}
Parameter buffer, cf. \ref{PCLINK:PARBUF}

\paragraph{Return value}

\subsection{0x0C - REMOVE Remove file }
\paragraph{Parameters}
Parameter buffer, cf. \ref{PCLINK:PARBUF}

\paragraph{Return value}

\subsection{0x0D - CHMOD Change file permissions }
\paragraph{Parameters}
Parameter buffer, cf. \ref{PCLINK:PARBUF}

\paragraph{Return value}

\subsection{0x0E - MKDIR Create a directory }
\paragraph{Parameters}
Parameter buffer, cf. \ref{PCLINK:PARBUF}

\paragraph{Return value}

\subsection{0x0F - RMDIR Remove directory }
\paragraph{Parameters}
Parameter buffer, cf. \ref{PCLINK:PARBUF}

\paragraph{Return value}

\subsection{0x10 - CHDIR Change the current working directory }
\paragraph{Parameters}
Parameter buffer, cf. \ref{PCLINK:PARBUF}

\paragraph{Return value}

\subsection{0x11 - GETCWD Return the current working directory }
\paragraph{Parameters}
Parameter buffer, cf. \ref{PCLINK:PARBUF}

\paragraph{Return value}

\subsection{0x13 - DFREE Return information about unit }
\paragraph{Parameters}
Parameter buffer, cf. \ref{PCLINK:PARBUF}

\paragraph{Return value}

\subsection{0x14 - CHVOL Change the current volume / drive }
\paragraph{Parameters}
Parameter buffer, cf. \ref{PCLINK:PARBUF}

\paragraph{Return value}

\part{Smartdevice}
\section{Introduction}
Smartdevice always uses the device id 0x45 (P6: ?)

\section{Command overview}
\begin{tabular}[h!]{l|l|l|l}
Cmd & Name & Parameters & Return value \\ \hline 
0x55 & Submit URL & AUX = URL length, data frame with URL & Success or failure \\
0x93 & Send datetime & None & Date and Time
\end{tabular}

\section{Details}
\subsection{0x55 - Submit URL }
This command submits a URL for opening by the host systems default application for the URL's scheme.

\paragraph{Parameters}
AUX 1 contains the LSB of the URL length, while AUX 2 contains the MSB of the URL length.
The client sends a data frame containing the URL upon command ACK.

\paragraph{Return value}
If the URL is longer than 2000 bytes, the server answers with a command NAK.
Otherwise, the server answers with a command ACK.

\subsection{0x93 - Send date and time}
This command requests the date and time from the server.
The client then sets the date and time on the Atari.

\paragraph{Parameters}
None

\paragraph{Return value}
The server returns a dataframe containing 6 bytes with the recent date and time as follows:\\

\begin{tabular}{l|l|l}
Index & Size & Meaning \\ \hline
0 & Byte & Day of month \\
1 & Byte & Month \\
2 & Byte & Year without century \\
3 & Byte & Hours \\
4 & Byte & Minutes \\
5 & Byte & Seconds
\end{tabular}


%%\bibliographystyle{plain}
%%\bibliography{references}
\end{document}
